\chapter{Questions de cours}
\thispagestyle{plain} % Supprimer le header & le footer sur cette page en laissant la numérotation
\newpage

%----------------------------------------------------------------------------------------
%	Matériaux
%----------------------------------------------------------------------------------------

\section{Matériaux}

%--------------------------------------------

\subsection{Les Fontes}

\question{Qu'est ce qu'une fonte ? Quelles sont ses caractéristiques ? Citer au moins un exemple de fonte ? Quels sont ses avantages et inconvéniant ?}{
\begin{itemize}
\item Alliage de fer et de carbone, contenant 2 à 5\% de carbone.
\item Module de Young (traction) : $E=83 000$ à $170 000 MPa$
\item Module de Coulomb (cisaillement) : $G=32 000$ à $66 000 MPa$
\item température de fusion : $T=1 150$ à $1 350 C$
\item Exemple : $EN-GJL 100$ Fontes à Graphite Lamellaire $Rm=100 MPa$\\
$EN-GJS 700-2$ Fontes à Graphite Sphéréoïdal $Rm=700 MPa$ et $A\%=2\%$
\item Avantages : Bonnes caractéristiques mécaniques (pour les fontes à graphite sphéroïdal); Amortissent les vibrations; Dures (\%C élevé); résistance à l'usure; bonne coulabilité (moulage); Grande inertie thermique.
\item Inconvéniants : Fragiles (non ductiles); Non soudables.
\end{itemize}}

%--------------------------------------------

\subsection{Les traitements Thermiques}

\question{Qu'est ce qu'une Trempe ?}{
\begin{itemize}
\item donne à l'acier une grande dureté par transformation de l'austénite en martensite.
\item trois phases principales :\\
1. Chauffage : destiné à amener l'acier à l'état austénitique.\\
2. Maintien à température d'austénisation qui dépend des dimensions et des formes de la pièce et aussi des types d'aciers.\\
3. Refroidissement (eau ou huile) : C'est lui qui conditionne la structure finale.
\end{itemize}}

\question{Qu'est ce qu'un revenu ?}{
\begin{itemize}
\item traitement complémentaire à la trempe. Il diminue ses effets, il supprime les tensions internes ; par conséquent, il atténue la fragilité mais également la dureté.
\item trois phases principales :\\
1. Chauffage : La température de revenu (dessous de la température d'austénisation)\\
2. Maintien : Le temps moyen est de 2 heures.\\
Refroidissement : Le temps idéal est d'une heure (trop court = transformation incomplète,; plus long = plus couteux).
\end{itemize}}

\question{Qu'est ce qu'un recuit ?}{
\begin{itemize}
\item Supprimer les tensions internes existant dans la pièce brute
\item Se pratique avec un chauffage, un maintien à température (30min), un refroidissement suffisamment lent à l'air ou au four pour éviter la formation de constituants de trempe.
\end{itemize}}

\question{Qu'est ce qu'une Cémentation ?}{
\begin{itemize}
\item sur des pièces en acier doux
\item formation d'une couche superficielle ayant absorbé du carbone et se transforme en acier dur.
\end{itemize}}

\question{Qu'est ce qu'une Nitrutration ?}{
\begin{itemize}
\item Obtenir une pièce résiliente à coeur et très dure en surface.
\item durcissement superficiel obtenu par réaction de l'azote et de certains alliages ferreux.
\end{itemize}}

\question{Qu'est ce qu'une Carbonituration ?}{
\begin{itemize}
\item procédé de durcissement superficiel permettant au métal d'absorber du carbone et de l'azote dans une atmosphère constituée de carbone et d'ammoniac
\item Le procédé est un compromis entre la cémentation et la nitruration.
\end{itemize}}

%----------------------------------------------------------------------------------------
%	Statique
%----------------------------------------------------------------------------------------

\section{Statique}

\subsection{Arc-boutement}

\question{Donner la notion d'arc-boutement. Quelle loi vérifie un solide arc-bouté ?}{L'arc-boutement est un phénomène de blocage qui peut survenir sur les guidages en translation.\\
Cf. Formule du cours}

%----------------------------------------------------------------------------------------
%	Hacheurs
%----------------------------------------------------------------------------------------

\section{Les hacheurs}

\question{Qu'est qu'un hacheur ?}{Les hacheurs sont des convertisseurs statiques de puissance reposant sur des dispositifs de commutation. Aucune pièce mécanique ne les compose d'où la dénomination "statique".}